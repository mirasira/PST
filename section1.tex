

\section{Kombinatorika} 
    \subsection{Pravidlo součinu} 
    \begin{itemize}
        \item Př. 15 chlapců, 13 dívek, kolik je možných tanečních párů?
        \item $15\cdot 13=195$
    \end{itemize}
    \subsection{Pravidlo součtu}
    \begin{itemize}
        \item Př. kolik je trojciferných čísel dělitelných pěti, obsahující 
        max jednu číslici 5
        \item $9+8+8\cdot 9 = 161$
    \end{itemize}
    \subsection{Permutace}
        \begin{itemize}
            \item Seřazení n prvků
            \item Př. abc, acb, bca, bac, cab, cba  $3\cdot 2\cdot 1=6$
        \end{itemize}
    \subsection{Variace}
        \begin{itemize}
            \item Uspořádaný výběr
            \item Př. abcd chci 2, je jich 12
            \item $\frac{n!}{(n-k)!}$
        \end{itemize}
    \subsection{Kombinace}
        \begin{itemize}
            \item Neuspořádaný výběr
            \item Př. mám 5 koček, chci 3 na dovolenou, 10 možností
            \item zápis ${5\choose 3}=\frac{n!}{k!(n-k)!}$
        \end{itemize}
    \subsection{Pascalův trojúhelník}
        \begin{tabular}{cccccccccccc}
            & & & & & &1& & & & & \\
            & & & & &1& &1& & & & \\
            & & & &1& &2& &1& & & \\
            & & &1& &3& &3& &1& & \\
            & &1& &4& &6& &4& &1& \\
            &1& &5& &10& &10& &5& &1
        \end{tabular}
    \subsection{Neuspořádaný výběr s opakováním}
        \begin{itemize}
            \item ${{n+k-1}\choose k}$
        \end{itemize}
    \subsection{Spárování}
        \begin{itemize}
            \item Př. a,b,c,d,e,f - $\frac{{6 \choose 2}{4 \choose 2}{2 \choose 2}}{3!}$
        \end{itemize}
\section{Klasická pravděpodobnost}
    \begin{enumerate}
        \item $\Omega$ množina všech výsledků stejně možných
        \item Laplaceova definice
            \begin{itemize}
                \item $P(a)=\frac{|A|}{|\Omega|} = \frac{\mbox{Počet příznivých}}{\mbox{Počet všech možných}}$
            \end{itemize}
        \item Komplementární (doplňkový) jev
            \begin{itemize}
                \item $A^c=\Omega \backslash A$
                \item $P(A^c)=1-P(A)$
            \end{itemize}
    \end{enumerate}
\section{Geometrická pravděpodobnost} 
\section{Kolmogorova pravděpodobnost}
\section{Bayesova věta}
    \begin{enumerate}
        \item Podmíněná pravděpodobnost
            \begin{itemize}
                \item $P(A|B)=\frac{P(A\cap B)}{P(B)}$
            \end{itemize}
        \item Věta o úplné pravděpodobnosti
            \begin{itemize}
                \item $P(A)=\sum P(A|B_i)P(B_i)$
            \end{itemize}
        \item Bayesova věta
            \begin{itemize}
                \item $P(B|A)=\frac{P(A|B)P(B)}{P(A)}$
            \end{itemize}
    \end{enumerate}

\section{Náhodná veličina}
    \subsection{Střední hodnota}
        \begin{itemize}
            \item $\mathbb{E}X = \sum_{i=1}^\infty x_i \cdot P(X=x_i)$
            \item $\mathbb{E}X = \int^\infty_{-\infty} xf(x)dx$
        \end{itemize}
    \subsection{Moment}
        Je bad\\
        $\mathbb{E}|X-\mathbb{E}|$
    \subsection{Rozptyl}
        $\var X = \mathbb{E}(X-\mathbb{E})^2$
    \subsection{Kovariance}
        $\cov(X,Y)=\mathbb{E}(X-\mathbb{E}X)(Y-\mathbb{E}Y)$
        $\cov(X,X)=\var(X)$
    \subsection{Čebyševova nerovnost}
    
\newpage

Př:
Kolikrát musíme hodit kostkou, aby pst, že odchylka podílu sudých čísel od $\frac{1}{2}$
byla maximálně 0,05 s pravděpodobností alespoň 0,95?\\
$X$ \dots podíl sudých čísel v n hodech kostkou\\
$X_i$ \dots podíl sudých čísel v i-tém hodu, neboli\\
$X_i = 1$, pokud v i-tém hodu padne sudé číslo\\
$X_i = 0$, pokud v i-tém hodu padne liché číslo

        
